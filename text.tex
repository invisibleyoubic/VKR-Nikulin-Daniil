\documentclass[30pt]{article}
\usepackage[utf8]{inputenc}
\usepackage[english,russian]{babel}
\usepackage[left=1cm, right=1cm, top=1cm, bottom=2cm]{geometry} % поля страницы


\begin{document}
\textbf{1 слайд}

Здравствуйте, я Никулин Даниил, тема моей выпускной квалификационной работы называется "Разработка и оптимиация оценочной функции для учета межмолекулярных взаимодействий в белковых комплексах". Научный руководитель -- Полуян Сергей Владимирович, рецензент -- Белов Олег Валерьевич.

\vspace{+5mm}


\textbf{2 слайд}

Функцию для оценки энергии взаимодействия лиганда с белком в заданной пространственной конфигурации называют <<оценочной функцией>>. В настоящее время разработано множество оценочных функций, которые подразделяются на группы, исходя из принципов их построения. Помимо точности оценки искомой энергии взаимодействия важными критерием выбора оценочной функции является вычислительная сложность процедуры оценки, поэтому при моделировании взаимодействия на больших временных масштабах прибегают к моделям с упрощенным представлением белков, а также исключают конформационную подвижность, рассматривая компоненты комплекса как <<твёрдые>> тела, совершающие в растворителе только поступательные и вращательные движения.

Целями работы являются: разработка, реализация, оптимизация и~верификация оценочной функции для учета межмолекулярных взаимодействий в~белковых комплекса в~рамках библиотеки для полноатомного моделирования белковых комплексов PSM

Разработанная оценочная функция состоит из трех слагаемых

\vspace{+5mm}


\textbf{3 слайд}

Первое слагаемое оценочной функции - потенциал Леннард-Джонса

Указанные парметры получены из файла топологии для соотетвующих типов атомов в рассматриваемой паре атомов с индексами i и j.

Стоит отметить, что в CHARMM для описания сил Ван-дер-Ваальса сейчас используется двойной экспоненциальный потенциал. Он позовляет более точно оценить энергию, поскольку использует отдельные функции для оценки эффекта притяжения и отталкивания, а также отдельную процедуру для учета дальнодействующих взаимодействий. Этот учет нужен в молекулярной динамике. В нашем исследовании достаточно точности потенциала 6-12.

На рисунке приведены значения потенциала в~зависимости от расстояниям между двумя атомами: углерода ($\varepsilon=0.11, R/2 = 2$) и~водорода ($\varepsilon=0.031, R/2 = 1.25$).

\vspace{+5mm}


\textbf{4 слайд}

Второе слагаемое оценочной функции - Потенциал Кулона

Потенциал кулона рассматривается в классическом виде, где присутсвуют два множителя, дополнительный множитель это квадратная функция, она нужна для сглаживания кулоновского потенциала и для коэффициента отсечения.

На графике виден эффект применения дополнительного множителя.

Атомные заряды получены с помощью программы PDB2PQR, вместо первой дроби используется константа, применяемая в силовом поле CHARMM, равная 332.0716. Последний множитель с коэффициентом к = 14А определяет радиус сферы взаимодействия, то есть, для $d_{ij} > k$ вычисления не производятся

\vspace{+5mm}


\textbf{5 cлайд}

Последнее слагаемое оценочной функции -- модель неявного растворителя, которая представляет собой модель Гауссовских Гидратных оболочек

При вычислении первой суммы в выражении используется заранее вычисленные при температуре 298.15К значения полной энергии гидратации атомов $\Delta G_i$. Перечисленные выше праметры представлены в таблице и разбиты по группам для каждого тип атома. При реализации оценочной функции использовался принцип определения типов атомов, такой же, как и в силовом поле CHARMM. Данная модель тоже описывает электро-статические взаимодействия

\vspace{+5mm}


\textbf{6 cлайд}

Оценочная функция была реализована на языке программирования~С++ и~добавлена в~проект PSM в~виде набора модулей. Для нахождения взаимодействующих атомов был реализован алгоритм прямого перебора всех возможных пар атомов.

В~случае фиксированного количества атомов в~каждой цепи временная сложность алгоритма прямого перебора возрастает с представленной на слайде ассимптотикой. Стоит отметить, что в случае различного числа атомов в~цепях асимптотика прямого перебора также останется квадратичной.

Для ускорения процесса нахождения взаимодействующих пар атомов использовалась структура данных k-d-дерево. Сложность вычисления с использованием kd дерева представлена на слайде. Следует отметить, построенное k-d-дерево обладает фиксированной пространственной сложностью.

\vspace{+5mm}

\newpage

\textbf{7 слайд}

На первом рисунке представлено время поиска взаимодействующих пар атомов для двух блековых комлексов с разным количеством атомов.
Обозначения на рисунке: nn -- алгоритм прямого перебора, rosetta -- программный комплекс розетта, kd -- алгоритм с построением kd дерева при каждом поиске, kd opt -- алгоритм с предварительным построением kd дерева.

Как видно из результатов, алгоритм с~применением прямого перебора всех атомов работает в~несколько раз медленнее других подходов. Однако, при сравнении с~Rosetta, несмотря на то, что коэффициент ускорения присутствует и~больше единицы, получаемое ускорение незначительно. Это связано с~тем, что в~Rosetta реализован собственный механизм построения карты взаимодействия атомов, который, в том числе, использует структуру данных k-d-дерево.

На втором рисунке продемонстрированы коэффициенты ускорения соответствующие времени поиска взаимодействующих атомов разработанной функции в~отношении с другими алгоритмами. Слева представлено получаемое ускорение при использовании алгоритма kd opt по сравнению с алгоритмом прямого перебора, а справа по сравнению с розеттой.

\vspace{+5mm}


\textbf{8 слайд}

На рисунках приведены результаты численных оценок для 84~комплексов. Средствами пакета CHARMM выполнена оценка энергии для представленных в~разработанной оценочной функции компонент. Для полученных значений рассчитан линейный коэффициент корреляции Пирсона.

На первом рисунке слева представлены результаты численного эксперимента для потенциала Леннард-Джонса, а справа для потенциала Кулона.

На втором рисунке слева представлены результаты численного эксперимента для неявного растворителя, а справа для оцени энергии связывания.

\vspace{+5mm}


\textbf{9 слайд}

На рисунке для тестового набора белков продемонстрировано сравнение оценок энергии связывания, которые были вычислены с~помощью разработанной оценочной функции (обозначено $F_s$). Для наглядности посчитана линейная регрессия и~коэффициент корреляции Пирсона.

Полная энергия связывания включает в~себя все три компонента (включая растворитель) и~вычисляется как разница между оценкой энергии комплекса в~связанном состоянии и~оценками энергий в~свободном состоянии, т.е.~когда компоненты комплекса друг с~другом не взаимодействуют. Представленная на рисунке оценка энергии связывания $F_s$ определяется по формуле, представленной на слайде.

\vspace{+5mm}


\textbf{10 слайд}

В результате выполненной работы достигнуты следующие результаты.

\begin{enumerate}
	\item Разработана и реализована оценочная функция с~использованием набора параметров силового поля CHARMM в~рамках библиотеки PSM.
	\item Выполнена оптимизация процедуры поиска взаимодействующих пар атомов с~помощью применения структуры данных k-d-дерево.
	\item Проведены различные численные эксперименты, демонстрирующие приемлемую высокую корреляцию оценок с~результатами силовых полей CHARMM и~Rosetta.
	\item На примере двух белков продемонстрировано преимущество применения структуры данных k-d-дерево для поиска взаимодействующих пар атомов.
\end{enumerate}

Результаты работы представлялись на всероссийской конференции~<<Информационно-телекоммуникационные технологии и математическое моделирование высокотехнологичных систем 2023>>, которая проходила 17-21~апреля~2023 году в~Москве.

Разработанная оценочная функция используется в~рамках программного пакета PSM, позволяет гибко настраивать все параметры потенциалов и~быстро выполнять оценку энергии взаимодействия в~комплексах.

\end{document} 
