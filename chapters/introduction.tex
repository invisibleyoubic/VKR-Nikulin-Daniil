\newpage
\begin{center}
  \textbf{\large АННОТАЦИЯ}
\end{center}

В настоящей работе рассматриваются белковые комплексы вида белок-белок, где в качестве компонент комплекса выступают белки, представленные в полноатомном виде. При моделировании процесса образования устойчивого комплекса компонентами при их нековалентном взаимодействии друг с другом возникает необходимость в вычислении энергии такого взаимодействия. Одним из существующих методов для вычисления энергии взаимодействия является использование оценочной функции, которая для заданной пространственной конфигурации компонент позволяет приближенно оценить искомую энергию. В работе приведено описание разработанной оценочной функции и ее оптимизированной при помощи Kd-дерева модификации, в которых учитываются силы межатомных взаимодействий, представленные эмпирическими потенциалами Кулона и Леннард-Джонса. Молекулы растворителя в явном виде не рассматриваются, для этого энергия сольватации вычисляется в рамках модели неявного растворителя. Для тестового набора комплексов приведены результаты численных экспериментов, демонстрирующие ускорение вычисления энергии взаимодействия без потери точности, полученные с помощью разработанной оптимизированной и неоптимизированной функций. (Представлены результаты сравнения времени выполнения расчетов разработанной и оптимизированной функцией с существующими методами). Выполненная реализация оптимизированой оценочной функции представлена в рамках программного пакета для полноатомного моделирования белка и будет использоваться в экспериментах по оценке константы скорости связывания в белковых комплексах с помощью кинетического метода Монте-Карло. Разработанная оценочная функция обладает набором параметров, выбор которых позволит различным образом рассматривать процесс формирования комплекса.

\onehalfspacing
\setcounter{page}{2}

\newpage
\renewcommand{\contentsname}{\centerline{\large СОДЕРЖАНИЕ}}
\tableofcontents

\newpage
\begin{center}
  \textbf{\large ВВЕДЕНИЕ}
\end{center}
\addcontentsline{toc}{chapter}{ВВЕДЕНИЕ}

\textbf{Актуальность}

В настоящее время разработано множество оценочных функций, которые подразделяются на группы, исходя из принципов их построения. Например, распространено нестрогое деление на эмпирические, статистические и функции на основе силовых полей. Помимо точности оценки искомой энергии взаимодействия важными критерием выбора оценочной функции является вычислительная сложность процедуры оценки, поэтому при моделировании взаимодействия на больших временных масштабах прибегают к моделям с упрощенным представлением белков, а также исключают конформационную подвижность, рассматривая компоненты комплекса как <<твёрдые>> тела, совершающие в растворителе только поступательные и вращательные движения.

\textbf{Цель выпускной квалификационной работы} 

Целями настоящей работы являются: разработка, реализация и оптимизация оценочной функции для учета межмолекулярных взаимодействий в белковых комплексах. Функция разработана для оценки энергии при моделировании процесса образования белкового комплекса с помощью кинетического метода Монте-Карло. В основе метода лежит классическая теория переходного состояния, где в процессе моделирования система движется в сторону наименьшей полной энергии по пути с наименьшими энергетическими барьерами, что позволяет модельной системе на пути к термодинамическому равновесию проходить через последовательность квазиравновесных состояний. Следует отметить, что для моделирования процесса образования комплекса достаточно сформировать оценочную функцию, учитывающую только парные межатомные взаимодействия и влияние растворителя. 

\textbf{Задачи выпускной квалификационной работы:}

ПОКА НЕ ЗНАЮ. Возможно, нужно перенести из целей в задачи первое предложение

\textbf{Научной новизной обладают следующие результаты выпускной
  квалификационной работы:}

ТОЖЕ ПОКА НЕ ЗНАЮ
