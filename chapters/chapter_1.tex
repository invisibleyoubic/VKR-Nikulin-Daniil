\newpage
\begin{center}
  \textbf{\large 1. ТЕОРЕТИЧЕСКАЯ ЧАСТЬ}
\end{center}
\refstepcounter{chapter}
\addcontentsline{toc}{chapter}{1. ТЕОРЕТИЧЕСКАЯ ЧАСТЬ}


\section{Потенциал Кулона}

Потенциал Кулона - это одна из основных концепций в физике, которая описывает взаимодействие заряженных частиц. Он является мерой энергии, необходимой для перемещения заряда от бесконечности до некоторой точки в пространстве, где располагается заряд, который создает электрическое поле.

Если имеются два заряда:  $q_1$ и $q_2$, установленные на расстоянии $r$ друг от друга, то заряд $q_1$ оказывает воздействие на заряд $q_2$ с силой 
\[ C_y= \frac{q_1 q_2}{\pi \varepsilon r^2}, \] 
где:
\begin{itemize}
	\item $\varepsilon$ - это константа, которая зависит от свойств среды, в которой расположены заряды
\end{itemize}

В данном случае, электрическое поле образовано зарядом $q_1$, и потенциал Кулона этого поля в данной точке – это энергия, необходимая для перемещения заряда из бесконечности (где потенциал равен нулю) в данную точку. Он рассчитывается по формуле: 
\[ V = \frac{k Q}{r}, \]
где:
\begin{itemize}
	\item $k = \frac{1}{4 \Pi \varepsilon}$, $Q$ - заряд
	\item $r$ - расстояние между точкой и зарядом $Q$.
\end{itemize}

Знание потенциала Кулона важно для понимания многих явлений в электродинамике и физике зарядов. Например, знание потенциала Кулона может помочь в расчете энергии, необходимой для работы с зарядами или для определения действия различных сил в заряженных частицах.

Для решения многих физических задач, в которых участвуют заряды, часто используются линии уровня потенциала Кулона, которые изображаются в виде изолиний на графиках, что позволяет проще визуализировать электростатическую картину взаимодействия.

% Потом уберу - без этого не соберется бибтех
Тестовое упоминание источника \cite{666.666}


\section{Потенциал Леннард-Джонса}

Потенциал Леннард-Джонса - это математическая функция, которая используется для описания взаимодействия между частицами в некоторых моделях молекулярной динамики и статистической физики. Он был назван в честь физиков Джона Леннарда и Джона Джонса.

Функция потенциала Леннарда-Джонса зависит от расстояния между двумя частицами $r$ и имеет вид:
\[ V(r) = 4\varepsilon \left[ \left( \frac{\sigma}{r} \right)^{12} - \left( \frac{\sigma}{r} \right)^6 \right], \]
где:
\begin{itemize}
	\item $\varepsilon$ - это глубина потенциальной ямы
	\item $\sigma$ - это расстояние между частицами, при котором потенциальная энергия равна нулю.
\end{itemize}

Характерной особенностью этой функции является наличие двух слагаемых. Первое слагаемое описывает отталкивание между атомами при очень близких расстояниях ( $r < \sigma$ ), и имеет форму $\frac{\sigma^{12}}{r^{12}}$. Второе слагаемое описывает притяжение между атомами на более дистанционных расстояниях ( $r > \sigma$ ), и имеет форму $(-2\varepsilon) * \frac{\sigma^6)}{r^6}$.

У потенциала Леннарда-Джонса есть несколько важных свойств. Он является ограниченной функцией, т.е. у него есть минимальное значение, которое достигается при определенном значении расстояния между частицами. Это обуславливает специфические структуры, которые формируются в жидкостях, плохо растворяющимися газах и других системах в зависимости от параметров потенциала Леннарда-Джонса.

Потенциал Леннарда-Джонса используется в молекулярной динамике для симуляции взаимодействия между частицами жидкостей, газов и твердых тел. Это позволяет исследовать свойства систем, в которых используются молекулярные и атомные масштабы. Одним из наиболее важных приложений потенциала Леннарда-Джонса является моделирование систем белков и РНК в молекулярной биологии.


\section{Неявный растворитель}


Неявный растворитель - это математическая модель, основанная на модели Гауссовских гидратных оболочек, которая используется для описания структуры гидратации белков и других макромолекул. Эта модель основана на предположении, что молекулы воды вблизи поверхности белка образовывают некоторую сферическую оболочку, в которой плотность молекул воды стремится к нулю. Разработчики использовали гауссовские функции, чтобы определить энергию гидратации каждого атома белка.

Гидратная оболочка атома - слой молекул воды, которые окружают атом и связаны с ним посредством водородных связей. То есть, гауссовские функции используются для моделирования гидратационной оболочки вокруг атомов белка, что позволяет более точно выполнить расчет энергии гидратации каждого отдельного атома.

Модель Гауссовских гидратных оболочек предполагает, что размещение молекул воды в области гидратации с увеличением расстояния от белка может быть описано функцией Гаусса. Эта функция характеризуется средним значением и стандартным отклонением, которые определяют размеры и форму гидратной оболочки.

Модель Гауссовских гидратных оболочек позволяет получить количественную оценку химических взаимодействий между белком и водой. В частности, эта модель используется для выявления ключевых аминокислотных остатков в белке, которые взаимодействуют с водой и играют важную роль в его структуре и функции.

Более новые версии модели, такие как модель гидратации Майкросольвентной оболочки или модель Гижи-Шавитца, учитывают сильные динамические и корреляционные эффекты между молекулами воды и их взаимодействие с белком.

Однако, важно понимать, что модель Гауссовских гидратных оболочек является лишь упрощенной моделью, которая не может учитывать полную динамику гидратации белка. Несмотря на это, она остается полезным инструментом для анализа структуры и функции белков в различных условиях.

Формула неявного растворителя имеет вид:
\[ E_s = \sum_i \Delta G_i + \sum_{i,j} \left( \frac{1}{2 \pi \sqrt{\pi}} \left[ -\Delta G_i e^{-\left( \tfrac{d_{ij} - {R_i}^{\textquotesingle}}{\lambda_i} \right)^2} - \Delta G_j e^{-\left( \tfrac{d_{ij} - {R_j}^{\textquotesingle}}{\lambda_j} \right)^2} \right] \right), \]
где:
\begin{itemize}
	\item $d_{ij}$ - евклидово расстояние между центрами атомов
	\item $\lambda_i$ и $\lambda_j$ - размеры гидратных оболочек атомов
	\item ${R_i}^{\textquotesingle}$ и ${R_j}^{\textquotesingle}$ - Ван-Дер-Ваальсовы радиусы атомов, которые соответствуют половине расстояния в потенциале Леннард-Джонса
	\item $\Delta G_i$ и $\Delta G_j$ - энергии гидратации в зависимости от типа атома
\end{itemize}
