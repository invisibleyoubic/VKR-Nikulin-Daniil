\newpage
\begin{center}
  \textbf{\large ЗАКЛЮЧЕНИЕ}
\end{center}
\refstepcounter{chapter}
\addcontentsline{toc}{chapter}{ЗАКЛЮЧЕНИЕ}


В результате выполненной работы достигнуты следующие результаты.
\vspace{-10pt}
\begin{enumerate}
	\setlength{\itemsep}{0pt}
	\setlength{\parskip}{0pt}	
	\item Разработана и реализована оценочная функция с~использованием набора параметров силового поля CHARMM в~рамках библиотеки PSM. 
	\item Выполнена оптимизация процедуры поиска взаимодействующих пар атомов с~помощью применения структуры данных k-d-дерево.
	\item Проведены различные численные эксперименты, демонстрирующие приемлемую высокую корреляцию оценок с~результатами силовых полей CHARMM и~Rosetta.
	\item На примере двух белков продемонстрировано преимущество применения структуры данных k-d-дерево для поиска взаимодействующих пар атомов.
\end{enumerate}
\vspace{-10pt}

Результаты работы представлялись на всероссийской конференции~<<Информационно-телекоммуникационные технологии и математическое моделирование высокотехнологичных систем 2023>>, которая проходила 17-21~апреля~2023 году в~Москве. Тезисы доклада представлены в~работе~\cite{rudn}.

Задача разработки и~реализации оценочной функции поставлена моим научным руководителем, который также является руководителем научно-исследовательской работы проводимой в~университете. Разработанная оценочная функция используется в~рамках программного пакета PSM, позволяет гибко настраивать все параметры потенциалов и~быстро выполнять оценку энергии взаимодействия в~комплексах. 

Следует отменить, что несмотря на то, что применение структуры данных k-d-дерево позволило в~несколько раз уменьшить временную сложность поиска атомов даже в~случае большого количества взаимодействующих атомов, время поиска попадает в~довольной широкий диапазон значений. Указанное отклонение можно значительно уменьшить с~помощью использования других структур данных. Применение других структур данных для поиска взаимодействующих атомов может стать темой дальнейших исследований.